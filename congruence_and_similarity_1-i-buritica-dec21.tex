\documentclass{beamer}
\usepackage[inline]{asymptote}
\usetheme{default}
\def\asydir{asy}
\title{I --- Congruence and Similarity}
\author{Andres Buritica}
\date{December 8, 2021}
\begin{document}
\begin{asydef}
  import olympiad;
  markscalefactor=1;
\end{asydef}
\begin{frame}[fragile]
  \titlepage{}
\end{frame}
\section{Intro}
  \subsection{Congruence}
    \begin{frame}[fragile]
      \frametitle{Congruence}
      We say that triangles $ABC$ and $XYZ$ are \emph{congruent} if \[BC=YZ,\ CA=ZX,\
      AB=XY,\ \angle BAC=\angle YXZ,\]
      \[\angle ABC=\angle XYZ,\ \angle ACB=\angle XZY.\]

      If all of these conditions are true then we can write \[\triangle
      ABC\cong\triangle XYZ.\] We also know that $|ABC| = |XYZ|$.
      
      \begin{center}
        \begin{asy}
          pair A=(13,73), B=(0,0), C=(100,0);
          pair offset=(150,0);
          pair X=A+offset, Y=B+offset, Z=C+offset;
          label("$A$",A,N);
          label("$B$",B,SW);
          label("$C$",C,SE);
          label("$X$",X,N);
          label("$Y$",Y,SW);
          label("$Z$",Z,SE);
          draw(A--B--C--A);
          draw(X--Y--Z--X);
          add(pathticks(A--B,1,.5,3,8));
          add(pathticks(X--Y,1,.5,3,8));
          add(pathticks(B--C,2,.5,3,8));
          add(pathticks(Y--Z,2,.5,3,8));
          add(pathticks(C--A,3,.5,3,8));
          add(pathticks(Z--X,3,.5,3,8));
          draw(anglemark(B,A,C));
          draw(anglemark(Y,X,Z));
          draw(anglemark(A,C,B,8,10,12));
          draw(anglemark(X,Z,Y,8,10,12));
          draw(anglemark(C,B,A,8,10));
          draw(anglemark(Z,Y,X,8,10));
        \end{asy}
      \end{center}
    \end{frame}
    \begin{frame}
      \frametitle{Congruence tests}
      \begin{itemize}
        \item A \emph{congruence test} is a set of properties that uniquely
          specify the triangle \pause
        \item Example: construct triangles with $BC=10$, area 20 and $\angle
          BAC=60^\circ$. \pause
        \item Then if two triangles have the same properties, they must be the
          same i.e.\ be congruent \pause
        \item For instance, we've just shown that if two triangles have the same
          area, a side of the same length, and the same angle opposite that
          side then they must be congruent \pause
        \item Though I have seen that test used in a problem before, what
          follows are the canonical congruence tests that are more likely
          to come up
      \end{itemize}
    \end{frame}
    \begin{frame}[fragile]
      \frametitle{SSS}
      If $BC=YZ,\ CA=ZX,\ AB=XY$ then $\triangle ABC\cong\triangle XYZ$.
      \begin{center}
        \begin{asy}
          pair A=(13,73), B=(0,0), C=(100,0);
          pair offset=(150,0);
          pair X=A+offset, Y=B+offset, Z=C+offset;
          label("$A$",A,N);
          label("$B$",B,SW);
          label("$C$",C,SE);
          label("$X$",X,N);
          label("$Y$",Y,SW);
          label("$Z$",Z,SE);
          draw(A--B--C--A);
          draw(X--Y--Z--X);
          add(pathticks(A--B,1,.5,3,8));
          add(pathticks(X--Y,1,.5,3,8));
          add(pathticks(B--C,2,.5,3,8));
          add(pathticks(Y--Z,2,.5,3,8));
          add(pathticks(C--A,3,.5,3,8));
          add(pathticks(Z--X,3,.5,3,8));
        \end{asy}
      \end{center}
    \end{frame}
    \begin{frame}[fragile]
      \frametitle{SAS}
      If $AB=XY,\ \angle ABC=\angle XYZ,\ BC=YZ$ then $\triangle
      ABC\cong\triangle XYZ$.
      \begin{center}
        \begin{asy}
          pair A=(13,73), B=(0,0), C=(100,0);
          pair offset=(150,0);
          pair X=A+offset, Y=B+offset, Z=C+offset;
          label("$A$",A,N);
          label("$B$",B,SW);
          label("$C$",C,SE);
          label("$X$",X,N);
          label("$Y$",Y,SW);
          label("$Z$",Z,SE);
          draw(A--B--C--A);
          draw(X--Y--Z--X);
          add(pathticks(A--B,1,.5,3,8));
          add(pathticks(X--Y,1,.5,3,8));
          add(pathticks(B--C,2,.5,3,8));
          add(pathticks(Y--Z,2,.5,3,8));
          draw(anglemark(C,B,A));
          draw(anglemark(Z,Y,X));
        \end{asy}
      \end{center}
    \end{frame}
    \begin{frame}[fragile]
      \frametitle{AAS}
      If $\angle ABC=\angle XYZ,\ \angle BCA=\angle YZX,\ BC=YZ$ then $\triangle
      ABC\cong\triangle XYZ$.
      \begin{center}
        \begin{asy}
          pair A=(13,73), B=(0,0), C=(100,0);
          pair offset=(150,0);
          pair X=A+offset, Y=B+offset, Z=C+offset;
          label("$A$",A,N);
          label("$B$",B,SW);
          label("$C$",C,SE);
          label("$X$",X,N);
          label("$Y$",Y,SW);
          label("$Z$",Z,SE);
          draw(A--B--C--A);
          draw(X--Y--Z--X);
          add(pathticks(B--C,1,.5,3,8));
          add(pathticks(Y--Z,1,.5,3,8));
          draw(anglemark(C,B,A));
          draw(anglemark(Z,Y,X));
          draw(anglemark(A,C,B,8,10));
          draw(anglemark(X,Z,Y,8,10));
        \end{asy}
      \end{center}
    \end{frame}
    \begin{frame}[fragile]
      \frametitle{SSA?}
      If $AB=XY,\ BC=YZ,\ \angle BCA=\angle YZX$ then you don't necessarily know
      $\triangle ABC\cong\triangle XYZ$.
      \begin{center}
        \begin{asy}
          pair A=(13,73), B=(0,0), C=(100,0);
          path circ1=Circle(B,length(A-B));
          pair offset=(150,0);
          pair X, Y=B+offset, Z=C+offset;
          X=intersectionpoints(circ1,A--C)[0]+offset;
          label("$A$",A,N);
          label("$B$",B,SW);
          label("$C$",C,SE);
          label("$X$",X,N);
          label("$Y$",Y,SW);
          label("$Z$",Z,SE);
          draw(A--B--C--A);
          draw(X--Y--Z--X);
          add(pathticks(B--C,2,.5,3,8));
          add(pathticks(Y--Z,2,.5,3,8));
          add(pathticks(A--B,1,.5,3,8));
          add(pathticks(X--Y,1,.5,3,8));
          draw(anglemark(A,C,B));
          draw(anglemark(X,Z,Y));
        \end{asy}
      \end{center}
    \end{frame}
    \begin{frame}[fragile]
      \frametitle{Fixed SSA}
      If $AB=XY,\ BC=YZ,\ \angle BCA=\angle YZX$ \textbf{and $AB>BC$} then
      $\triangle ABC\cong\triangle XYZ$.
      \begin{center}
        \begin{asy}
          pair A=(20,80), B=(0,0), C=(70,0);
          path circ1=Circle(B,length(A-B));
          pair offset=(150,0);
          pair X, Y=B+offset, Z=C+offset;
          X=intersectionpoints(circ1,A--C)[0]+offset;
          label("$A$",A,N);
          label("$B$",B,SW);
          label("$C$",C,SE);
          label("$X$",X,N);
          label("$Y$",Y,SW);
          label("$Z$",Z,SE);
          draw(A--B--C--A);
          draw(X--Y--Z--X);
          add(pathticks(B--C,2,.5,3,8));
          add(pathticks(Y--Z,2,.5,3,8));
          add(pathticks(A--B,1,.5,3,8));
          add(pathticks(X--Y,1,.5,3,8));
          draw(anglemark(A,C,B));
          draw(anglemark(X,Z,Y));
        \end{asy}
      \end{center}
    \end{frame}
    \begin{frame}[fragile]
      \frametitle{Fixed SSA}
      Why does this work? The other intersection is on the wrong side of $BC$.
      \begin{center}
        \begin{asy}
          pair A=(20,80), B=(0,0), C=(70,0);
          path circ1=Circle(B,length(A-B));
          real[] X=intersections(circ1,A,C);
          pair P=point(circ1,X[1]);
          label("$A$",A,N);
          label("$B$",B,SW);
          label("$C$",C,SW);
          label("$A'$",P,E);
          draw(circ1);
          draw(C--A--B--C--P);
          draw(anglemark(A,C,B));
        \end{asy}
      \end{center}
    \end{frame}
    \begin{frame}[fragile]
      \frametitle{RHS}
      This is a special case of fixed SSA\@.

      If $AB=XY,\ BC=YZ,\ \angle BCA=\angle YZX=90^\circ$ then $\triangle
      ABC\cong\triangle XYZ$.
      \begin{center}
        \begin{asy}
          pair A=(70,100), B=(0,0), C=(70,0);
          path circ1=Circle(B,length(A-B));
          pair offset=(150,0);
          pair X, Y=B+offset, Z=C+offset;
          X=intersectionpoints(circ1,A--C)[0]+offset;
          label("$A$",A,N);
          label("$B$",B,SW);
          label("$C$",C,SE);
          label("$X$",X,N);
          label("$Y$",Y,SW);
          label("$Z$",Z,SE);
          draw(A--B--C--A);
          draw(X--Y--Z--X);
          add(pathticks(B--C,2,.5,3,8));
          add(pathticks(Y--Z,2,.5,3,8));
          add(pathticks(A--B,1,.5,3,8));
          add(pathticks(X--Y,1,.5,3,8));
          draw(rightanglemark(A,C,B,5));
          draw(rightanglemark(X,Z,Y,5));
        \end{asy}
      \end{center}
    \end{frame}
  \subsection{Similarity}
    \begin{frame}[fragile]
      \frametitle{Similarity}
      We say that triangles $ABC$ and $XYZ$ are \emph{similar} if
      \[\frac{BC}{YZ}=\frac{CA}{ZX}=\frac{AB}{XY}=r,\]
      \[\angle ABC=\angle XYZ,\
      \angle BCA=\angle YZX,\ \angle CAB=\angle ZXY.\]

      If all of these conditions are true then we can write \[\triangle
      ABC\sim\triangle XYZ.\] We also know that $\frac{|ABC|}{|XYZ|}=r^2$.

      \begin{center}
        \begin{asy}
          pair A=(13,73), B=(0,0), C=(100,0);
          pair offset=(150,0);
          pair X=A/2+offset, Y=B/2+offset, Z=C/2+offset;
          label("$A$",A,N);
          label("$B$",B,SW);
          label("$C$",C,SE);
          label("$X$",X,N);
          label("$Y$",Y,SW);
          label("$Z$",Z,SE);
          draw(A--B--C--A);
          draw(X--Y--Z--X);
          add(pathticks(A--B,1,.5,3,8));
          add(pathticks(X--Y,1,.5,3,8));
          add(pathticks(B--C,2,.5,3,8));
          add(pathticks(Y--Z,2,.5,3,8));
          add(pathticks(C--A,3,.5,3,8));
          add(pathticks(Z--X,3,.5,3,8));
          draw(anglemark(B,A,C));
          draw(anglemark(Y,X,Z));
          draw(anglemark(A,C,B,8,10,12));
          draw(anglemark(X,Z,Y,8,10,12));
          draw(anglemark(C,B,A,8,10));
          draw(anglemark(Z,Y,X,8,10));
        \end{asy}
      \end{center}
    \end{frame}
    \begin{frame}[fragile]
      \frametitle{Similarity}
      Triangles can be directly similar or oppositely similar. For example,
      $\triangle ABC$ is directly similar to $\triangle XYZ$ and oppositely
      similar to $\triangle X'YZ$.

      We write: $\triangle ABC\overset +\sim\triangle XYZ$, $\triangle
      ABC\overset -\sim\triangle X'YZ$.

      \begin{center}
        \begin{asy}
          pair A=(13,73), B=(0,0), C=(100,0);
          pair offset=(150,30), t=(0.3,0.2);
          pair X=A*t+offset, Y=B*t+offset, Z=C*t+offset;
          pair W=2*foot(X,Y,Z)-X;
          label("$A$",A,N);
          label("$B$",B,SW);
          label("$C$",C,SE);
          label("$X$",X,N);
          label("$Y$",Y,SW);
          label("$Z$",Z,SE);
          label("$X'$",W,S);
          draw(A--B--C--A);
          draw(Z--X--Y--Z--W--Y);
        \end{asy}
      \end{center}
    \end{frame}
    \begin{frame}[fragile]
      \frametitle{PPP}
      If $\frac{BC}{YZ}=\frac{CA}{ZX}=\frac{AB}{XY}$ then $\triangle ABC\sim\triangle XYZ$.
      \begin{center}
        \begin{asy}
          pair A=(13,73), B=(0,0), C=(100,0);
          pair offset=(150,0);
          pair X=A/2+offset, Y=B/2+offset, Z=C/2+offset;
          label("$A$",A,N);
          label("$B$",B,SW);
          label("$C$",C,SE);
          label("$X$",X,N);
          label("$Y$",Y,SW);
          label("$Z$",Z,SE);
          draw(A--B--C--A);
          draw(X--Y--Z--X);
          add(pathticks(A--B,1,.5,3,8));
          add(pathticks(X--Y,1,.5,3,8));
          add(pathticks(B--C,2,.5,3,8));
          add(pathticks(Y--Z,2,.5,3,8));
          add(pathticks(C--A,3,.5,3,8));
          add(pathticks(Z--X,3,.5,3,8));
        \end{asy}
      \end{center}
    \end{frame}
    \begin{frame}[fragile]
      \frametitle{PAP}
      If $\frac{AB}{XY}=\frac{BC}{YZ},\ \angle ABC=\angle XYZ$ then $\triangle
      ABC\sim\triangle XYZ$.
      \begin{center}
        \begin{asy}
          pair A=(13,73), B=(0,0), C=(100,0);
          pair offset=(150,0);
          pair X=A/2+offset, Y=B/2+offset, Z=C/2+offset;
          label("$A$",A,N);
          label("$B$",B,SW);
          label("$C$",C,SE);
          label("$X$",X,N);
          label("$Y$",Y,SW);
          label("$Z$",Z,SE);
          draw(A--B--C--A);
          draw(X--Y--Z--X);
          add(pathticks(A--B,1,.5,3,8));
          add(pathticks(X--Y,1,.5,3,8));
          add(pathticks(B--C,2,.5,3,8));
          add(pathticks(Y--Z,2,.5,3,8));
          draw(anglemark(C,B,A));
          draw(anglemark(Z,Y,X));
        \end{asy}
      \end{center}
    \end{frame}
    \begin{frame}[fragile]
      \frametitle{AA}
      If $\angle ABC=\angle XYZ,\ \angle BCA=\angle YZX$ then $\triangle
      ABC\sim\triangle XYZ$.
      \begin{center}
        \begin{asy}
          pair A=(13,73), B=(0,0), C=(100,0);
          pair offset=(150,0);
          pair X=A/2+offset, Y=B/2+offset, Z=C/2+offset;
          label("$A$",A,N);
          label("$B$",B,SW);
          label("$C$",C,SE);
          label("$X$",X,N);
          label("$Y$",Y,SW);
          label("$Z$",Z,SE);
          draw(A--B--C--A);
          draw(X--Y--Z--X);
          draw(anglemark(C,B,A));
          draw(anglemark(Z,Y,X));
          draw(anglemark(A,C,B,8,10));
          draw(anglemark(X,Z,Y,8,10));
        \end{asy}
      \end{center}
    \end{frame}
    \begin{frame}[fragile]
      \frametitle{PPA?}
      If $\frac{AB}{XY}=\frac{BC}{YZ},\ \angle BCA=\angle YZX$ then you don't necessarily know
      $\triangle ABC\sim\triangle XYZ$.
      \begin{center}
        \begin{asy}
          pair A=(13,73), B=(0,0), C=(100,0);
          path circ1=Circle(B,length(A-B));
          pair offset=(150,0);
          pair X, Y=B/2+offset, Z=C/2+offset;
          X=intersectionpoints(circ1,A--C)[0]/2+offset;
          label("$A$",A,N);
          label("$B$",B,SW);
          label("$C$",C,SE);
          label("$X$",X,N);
          label("$Y$",Y,SW);
          label("$Z$",Z,SE);
          draw(A--B--C--A);
          draw(X--Y--Z--X);
          add(pathticks(B--C,2,.5,3,8));
          add(pathticks(Y--Z,2,.5,3,8));
          add(pathticks(A--B,1,.5,3,8));
          add(pathticks(X--Y,1,.5,3,8));
          draw(anglemark(A,C,B));
          draw(anglemark(X,Z,Y));
        \end{asy}
      \end{center}
    \end{frame}
    \begin{frame}[fragile]
      \frametitle{Fixed PPA}
      If $\frac{AB}{XY}=\frac{BC}{YZ},\ \angle BCA=\angle YZX$ \textbf{and $AB>BC$} then
      $\triangle ABC\sim\triangle XYZ$.
      \begin{center}
        \begin{asy}
          pair A=(20,80), B=(0,0), C=(70,0);
          path circ1=Circle(B,length(A-B));
          pair offset=(150,0);
          pair X, Y=B/2+offset, Z=C/2+offset;
          X=intersectionpoints(circ1,A--C)[0]/2+offset;
          label("$A$",A,N);
          label("$B$",B,SW);
          label("$C$",C,SE);
          label("$X$",X,N);
          label("$Y$",Y,SW);
          label("$Z$",Z,SE);
          draw(A--B--C--A);
          draw(X--Y--Z--X);
          add(pathticks(B--C,2,.5,3,8));
          add(pathticks(Y--Z,2,.5,3,8));
          add(pathticks(A--B,1,.5,3,8));
          add(pathticks(X--Y,1,.5,3,8));
          draw(anglemark(A,C,B));
          draw(anglemark(X,Z,Y));
        \end{asy}
      \end{center}
    \end{frame}
    \begin{frame}[fragile]
      \frametitle{RHS}
      This is a special case of fixed PPA\@.

      If $\frac{AB}{XY}=\frac{BC}{YZ},\ \angle BCA=\angle YZX=90^\circ$ then $\triangle
      ABC\sim\triangle XYZ$.
      \begin{center}
        \begin{asy}
          pair A=(70,100), B=(0,0), C=(70,0);
          path circ1=Circle(B,length(A-B));
          pair offset=(150,0);
          pair X, Y=B+offset, Z=C+offset;
          X=intersectionpoints(circ1,A--C)[0]+offset;
          label("$A$",A,N);
          label("$B$",B,SW);
          label("$C$",C,SE);
          label("$X$",X,N);
          label("$Y$",Y,SW);
          label("$Z$",Z,SE);
          draw(A--B--C--A);
          draw(X--Y--Z--X);
          add(pathticks(B--C,2,.5,3,8));
          add(pathticks(Y--Z,2,.5,3,8));
          add(pathticks(A--B,1,.5,3,8));
          add(pathticks(X--Y,1,.5,3,8));
          draw(rightanglemark(A,C,B,5));
          draw(rightanglemark(X,Z,Y,5));
        \end{asy}
      \end{center}
    \end{frame}
\section{Problems}
  \begin{frame}[fragile]
    \frametitle{Isosceles triangle}
    Let $ABC$ be a triangle. Prove that $AB=AC$ if and only if $\angle
    ABC=\angle ACB$. \pause
    \begin{center}
      \begin{asy}
        pair A=(50,30), B=(0,0), C=(100,0);
        label("$A$",A,N);
        label("$B$",B,SW);
        label("$C$",C,SE);
        draw(A--B--C--cycle);
      \end{asy}
    \end{center}
    \pause
    \begin{columns}
      \column{.5\textwidth}
        Going forward: \pause
        \begin{itemize}
          \item $AB=AC$ \pause
          \item $\triangle ABC\cong\triangle ACB$ (SSS) \pause
          \item $\angle ABC=\angle ACB$ \pause
        \end{itemize}
      \column{.5\textwidth}
        Going backward: \pause
        \begin{itemize}
          \item $\angle ABC=\angle ACB$ \pause
          \item $\triangle ABC\cong\triangle ACB$ (AAS) \pause
          \item $AB=AC$
        \end{itemize}
    \end{columns}
  \end{frame}
  \begin{frame}[fragile]
    \frametitle{Power of a point}
    Let $ABCD$ be a cyclic quadrilateral, and let $AC$ and $BD$ meet at $P$.
    Prove that $PA\times PC=PB\times PD$. \pause
    \begin{columns}
      \column{.5\textwidth}
        \begin{center}
          \begin{asy}
            pair A=(39,52), B=(39,-52), D=(-25,60), C=(-60,25);
            label("$A$",A,N);
            label("$B$",B,SE);
            label("$C$",C,W);
            label("$D$",D,NW);
            pair P=intersectionpoint(A--C,B--D);
            draw(A--B--C--D--cycle);
            draw(A--C);
            draw(B--D);
            label("$P$",P,SW);
            draw(Circle(origin,65));
          \end{asy}
        \end{center}
        \pause
      \column{.5\textwidth}
        \begin{itemize}
          \item $\triangle PAB\sim\triangle PDC$ (AA) \pause
          \item $\frac{PA}{PB}=\frac{PD}{PC}$ \pause
          \item $PA\times PC=PB\times PD$
        \end{itemize}
    \end{columns}
  \end{frame}
  \begin{frame}[fragile]
    \frametitle{Similar switch}
    Let $ABC$ and $ADE$ be triangles that are directly similar.
    Prove that $\triangle ABD$ and $\triangle ACE$ are also directly similar.

    \begin{columns}
      \column{.5\textwidth}
        \begin{center}
          \begin{asy}
            pair A=(0,0), B=(100,0), C=(80,20), D=(10,50);
            pair E=(C*D)/B;
            label("$A$",A,SW);
            label("$B$",B,SE);
            label("$C$",C,N);
            label("$D$",D,NE);
            label("$E$",E,W);
            draw(B--A--C);
            draw(D--A--E);
            draw(B--C,red);
            draw(D--E,red);
            draw(B--D,green);
            draw(C--E,green);
          \end{asy}
        \end{center}
        \pause
      \column{.5\textwidth}
        \begin{itemize}
          \item $\angle BAD=\angle CAE$ \pause
          \item $\triangle ABD\sim\triangle ACE$
        \end{itemize}
    \end{columns}
  \end{frame}
  \begin{frame}[fragile]
    \frametitle{Ice cream cone}
    Let $PA$ and $PB$ be tangents to a circle. Prove that $PA=PB$. \pause

    Construction: let $O$ be centre of the circle.
    \begin{columns}
      \column{.5\textwidth}
        \begin{center}
          \begin{asy}
            pair O=(0,0), A=(-30,-40), B=(30,-40);
            pair P=2*A*B/(A+B);
            label("$A$",A,SW);
            label("$B$",B,SE);
            label("$P$",P,S);
            label("$O$",O,N);
            draw(Circle(O,length(A)));
            draw(B--O--A--P--B--A);
            draw(P--O);
          \end{asy}
        \end{center}
        \pause
      \column{.5\textwidth}
        \begin{itemize}
          \item $\triangle PAO\cong\triangle PBO$ \pause
          \item $PA=PB$
        \end{itemize}
    \end{columns}
  \end{frame}
  \begin{frame}[fragile]
    \frametitle{Similar figures}
    Let points $ABCDWXYZ$ be such that $\triangle ABC\overset +\sim\triangle
    WXY$ and $\triangle BCD\overset +\sim\triangle XYZ$. Prove that $\triangle
    ABD\sim\triangle WXZ$ and $\triangle ACD\sim\triangle WYZ$.
    \begin{columns}
      \column{.5\textwidth}
        \begin{center}
          \begin{asy}
            pair A=(13,73), B=(0,0), C=(100,0), D=(50,20);
            pair x=(30,-20), y=(0,-0.5);
            pair W=A*y+x, X=B*y+x, Y=C*y+x, Z=D*y+x;
            draw(A--B--C--D--A--C);
            draw(B--D);
            draw(W--X--Y--Z--W--Y);
            draw(X--Z);
            label("$A$",A,N);
            label("$B$",B,SW);
            label("$C$",C,SE);
            label("$X$",X,NW);
            label("$Y$",Y,S);
            label("$Z$",Z,(N+NE)/2);
            label("$W$",W,E);
            label("$D$",D,(3*NW+SW)/4);
          \end{asy}
        \end{center}
        \pause
      \column{.5\textwidth}
        \begin{itemize}
          \item $\triangle ABD\sim\triangle WXZ$ \pause
          \item $\triangle ACD\sim\triangle WYZ$ \pause
        \end{itemize}
        In this case we say that $ABCD$ and $WXYZ$ are similar figures, written
        $ABCD\overset +\sim WXYZ$. \pause

        All of this still works if the triangles are instead oppositely similar.
    \end{columns}
  \end{frame}
  \begin{frame}[fragile]
    \frametitle{Menelaus}
    Let $ABC$ be a triangle. Let $X,Y,Z$ be collinear points on sides $BC,CA,AB$
    respectively.

    Prove that \[\frac{AZ}{ZB}\times\frac{BX}{XC}\times\frac{CY}{YA}=-1.\]\pause

    Construction: let $P,Q,R$ be the bases of the perpendiculars from $A,B,C$ to
    $XYZ$.
    \begin{columns}
      \column{.5\textwidth}
        \begin{center}
          \begin{asy}
            pair A=(13,73)*0.8, B=(0,0), C=(100,0)*0.8;
            pair X=0.7*C, Z=0.4*A;
            pair Y=extension(X,Z,A,C);
            label("$A$",A,N);
            label("$B$",B,SW);
            label("$C$",C,NE);
            label("$X$",X,S);
            label("$Y$",Y,N);
            label("$Z$",Z,SW);
            pair P=foot(A,X,Z), Q=foot(B,X,Z), R=foot(C,X,Z);
            draw(A--P--Y--A--B--C--R);
            draw(B--Q);
            label("$P$",P,W);
            label("$Q$",Q,NE);
            label("$R$",R,SW);
            draw(rightanglemark(A,P,Z,5));
            draw(rightanglemark(B,Q,X,5));
            draw(rightanglemark(C,R,Y,5));
          \end{asy}
        \end{center}
        \pause
      \column{.5\textwidth}
        \begin{itemize}
          \item $\frac{BX}{XC}=\frac{BQ}{CR}$ \pause
          \item $\frac{AZ}{ZB}\times\frac{BX}{XC}\times\frac{CY}{YA}=1$
            \pause
          \item Why is it -1?
        \end{itemize}
    \end{columns}
  \end{frame}
  \begin{frame}[fragile]
    \frametitle{Homothetic triangles}
    Let $ABC$ and $XYZ$ be two triangles such that $BC||YZ,\ CA||ZX,\ AB||XY$.
    Prove that $AX,\ BY$ and $CZ$ are concurrent.\pause

    Reverse reconstruction: let $P$ be the intersection of $BY$ and $CZ$. We
    prove $AX$ passes through $P$.
    \begin{columns}
      \column{.5\textwidth}
        \begin{center}
          \begin{asy}
            pair A=(13,73), B=(0,0), C=(100,0);
            pair X=A/2+(10,15), Y=B/2+(10,15), Z=C/2+(10,15);
            draw(A--B,red);
            draw(X--Y,red);
            draw(A--C,green);
            draw(X--Z,green);
            draw(B--C,blue);
            draw(Y--Z,blue);
            pair P=extension(B,Y,C,Z);
            draw(A--P,dashed);
            draw(B--P);
            draw(C--P);
            label("$A$",A,N);
            label("$B$",B,SW);
            label("$C$",C,SE);
            label("$X$",X,NE);
            label("$Y$",Y,S);
            label("$Z$",Z,S);
            label("$P$",P,S);
          \end{asy}
        \end{center}
        \pause
      \column{.5\textwidth}
        \begin{itemize}
          \item $ABCP\sim XYZP$ \pause
          \item $PAX$ collinear
        \end{itemize}
    \end{columns}
  \end{frame}
  \begin{frame}[fragile]
    \frametitle{Pythagoras}
    Let $ABC$ be a triangle such that $\angle BAC=90^\circ$. Prove that
    $AB^2+AC^2=BC^2$. \pause

    Construction: let $D$ be the base of the perpendicular from $A$ to $BC$.
    \begin{columns}
      \column{.5\textwidth}
        \begin{center}
          \begin{asy}
            pair A=(0,0), B=(60,0), C=(0,80);
            label("$A$",A,SW);
            label("$B$",B,SE);
            label("$C$",C,N);
            draw(rightanglemark(B,A,C,5));
            draw(A--B--C--cycle);
            pair D=intersectionpoint(B--C,A--(80,60));
            draw(A--D);
            label("$D$",D,NE);
            draw(rightanglemark(A,D,B,5));
          \end{asy}
        \end{center}
        \pause
      \column{.5\textwidth}
        \begin{itemize}
          \item $AB^2=BC\times BD$ \pause
          \item $AB^2+AC^2=BC^2$
        \end{itemize}
    \end{columns}
  \end{frame}
  \begin{frame}[fragile]
    \frametitle{Pythagoras}
    Let $ABC$ be a triangle such that $\angle BAC=90^\circ$. Prove that
    $AB^2+AC^2=BC^2$.

    Construction: let $D$ be the base of the perpendicular from $A$ to $BC$.
    \begin{columns}
      \column{.5\textwidth}
        \begin{center}
          \begin{asy}
            pair A=(0,0), B=(60,0), C=(0,80);
            label("$A$",A,SW);
            label("$B$",B,SE);
            label("$C$",C,N);
            draw(rightanglemark(B,A,C,5));
            draw(A--B--C--cycle);
            pair D=intersectionpoint(B--C,A--(80,60));
            draw(A--D);
            label("$D$",D,NE);
            draw(rightanglemark(A,D,B,5));
          \end{asy}
        \end{center}
      \column{.5\textwidth}
        Alternatively:\pause
        \begin{itemize}
          \item $\frac{|ABD|}{|ABC|}=\left(\frac{AB}{BC}\right)^2$\pause
          \item $AB^2+AC^2=BC^2$
        \end{itemize}
    \end{columns}
  \end{frame}
  \begin{frame}[fragile]
    \frametitle{Diagonals of a parallelogram}
    Let $ABCD$ be a parallelogram with $AB||CD$ and $BC||DA$. Let $AC$ intersect
    $BD$ at $P$, Prove that $P$ is the midpoint of $AC$ and of $BD$.
    \begin{columns}
      \column{.5\textwidth}
        \begin{center}
          \begin{asy}
            pair A=(0,0), B=(70,0), C=(100,30);
            pair D=C-B, P=C/2;
            draw(A--B--C--D--A);
            draw(A--C);
            draw(B--D);
            label("$A$",A,SW);
            label("$B$",B,SE);
            label("$C$",C,NE);
            label("$D$",D,N);
            label("$P$",P,(N+NE)/2);
          \end{asy}
        \end{center}
        \pause
      \column{.5\textwidth}
        \begin{itemize}
          \item $\triangle ABC\cong\triangle CDB$ \pause
          \item $\triangle ABP\cong\triangle CDP$ \pause
          \item $PA=PC,\ PB=PD$
        \end{itemize}
    \end{columns}
  \end{frame}
  \begin{frame}[fragile]
    \frametitle{Alternate segment switch}
    Let $A,B,C$ be collinear points, and let $D,E$ be points such that $AD=AE$.
    Prove that if $\angle ADB=\angle ACD$ then $\angle AEB=\angle ACE$. \pause

    \begin{columns}
      \column{.5\textwidth}
        \begin{center}
          \begin{asy}
            pair A=(0,0), B=(25,0), C=(100,0);
            pair D=(30,40), E=(40,-30);
            label("$A$",A,SW);
            label("$B$",B,SE);
            label("$C$",C,N);
            label("$D$",D,NE);
            label("$E$",E,S);
            draw(A--C--D--A--E--C);
            draw(D--B--E);
            add(pathticks(A--D));
            add(pathticks(A--E));
            draw(anglemark(D,C,A),red);
            draw(anglemark(A,D,B),red);
          \end{asy}
        \end{center}
        \pause
      \column{.5\textwidth}
        \begin{itemize}
          \item $\frac{AE}{AC}=\frac{AB}{AE}$ \pause
          \item $\angle AEB=\angle ACE$
        \end{itemize}
    \end{columns}
  \end{frame}
  \begin{frame}[fragile]
    \frametitle{Ptolemy}
    Let $ABCD$ be points in that order around a circle. Prove that $AB\times
    CD+BC\times AD=AC\times BD$. \pause

    Construction: let $P$ be the point on $AC$ such that $AB\times CD=AP\times
    BD$.

    \begin{columns}
      \column{.5\textwidth}
        \begin{center}
          \begin{asy}
            pair A=(39,52), B=(39,-52), D=(-25,60), C=(-60,25);
            label("$A$",A,N);
            label("$B$",B,SE);
            label("$C$",C,W);
            label("$D$",D,NW);
            pair P=A+(C-A)/length(C-A)*length(B-A)*length(D-C)/length(B-D);
            label("$P$",P,N);
            draw(A--B--P--A--D--C--B--D);
            draw(C--P);
            draw(circumcircle(A,B,C));
          \end{asy}
        \end{center}
        \pause
      \column{.5\textwidth}
        \begin{itemize}
          \item $\triangle ABP\sim\triangle DBC$ \pause
          \item $\triangle ABD\sim\triangle PBC$ \pause
          \item $AB\times CD+BC\times AD$ \\ $=AC\times BD$
        \end{itemize}
    \end{columns}
  \end{frame}
  \begin{frame}[fragile]
    \frametitle{Diameter of the incircle}
    Let $ABC$ be a triangle with incentre $I$ and $A$-excentre $I_A$. Let the
    incircle touch $BC$ at $D$ and the $A$-excircle touch $BC$ at $E$. Let $P$
    be the reflection of $D$ over $I$.

    Prove that $P$ is on $AE$. \pause
    
    Construction: let $X$ be the tangency point of the incircle on $AB$, and let
    $Y$ be the tangency point of the $A$-excircle on $AB$.
    \begin{columns}
      \column{.5\textwidth}
        \begin{center}
          \begin{asy}
            pair A=(13,73)*0.6, B=(0,0), C=(100,0)*0.6;
            pair I=incenter(A,B,C);
            pair IA=(B-A)*(C-A)/(I-A)+A;
            pair D=foot(I,B,C), Es=foot(IA,B,C);
            pair P=2*I-D, X=foot(I,A,B), Y=foot(IA,A,B);
            draw(Circle(I,length(D-I)));
            draw(Circle(IA,length(Es-IA)));
            draw(B--C);
            draw(A--IA--Es);
            draw(D--P);
            draw(IA--Y--A--foot(IA,A,C));
            draw(I--X);
            draw(A--Es, dashed);
            label("$A$",A,W);
            label("$B$",B,W);
            label("$C$",C,NE);
            label("$D$",D,SW);
            label("$I$",I,E);
            label("$I_A$",IA,SE);
            label("$E$",Es,N);
            label("$P$",P,N);
            label("$X$",X,W);
            label("$Y$",Y,W);
          \end{asy}
        \end{center}
        \pause
      \column{.5\textwidth}
        \begin{itemize}
          \item $AXIP\sim AYI_A E$ \pause
          \item $A,P,E$ collinear \pause
          \item Alternatively, triangles $PIX$ and $EI_A Y$ are homothetic
        \end{itemize}
    \end{columns}
  \end{frame}
  \begin{frame}[fragile]
    \frametitle{Symmedian}
    Let $ABC$ be a triangle with circumcircle $\Gamma$. Let the tangents to
    $\Gamma$ at $B$ and $C$ intersect at $P$, and let the midpoint of $BC$ be
    $M$. Prove that $\angle PAB=\angle MAC$.\pause

    Construction: let $O$ be the centre of $\Gamma$, and let $D$ be the base
    of the perpendicular from $A$ to $BC$.
    \begin{columns}
      \column{.5\textwidth}
        \begin{center}
          \begin{asy}
            pair A=(13,73)/1.1, B=(0,0), C=(100,0)/1.1;
            pair O=circumcenter(A,B,C);
            draw(circumcircle(A,B,C));
            pair M=(B+C)/2;
            pair P=length(B-O)*length(B-O)/(length(M-O)*length(M-O))*(M-O)+O;
            pair D=foot(A,B,C);
            draw(D--A--B--C--A--O--P--B);
            draw(M--A--P--C--O);
            label("$A$",A,NW);
            label("$B$",B,SW);
            label("$C$",C,SE);
            label("$D$",D,S);
            label("$O$",O,W);
            label("$M$",M,SE);
            label("$P$",P,E);
            draw(rightanglemark(P,M,B,5));
            draw(rightanglemark(O,C,P,5));
            draw(rightanglemark(A,D,B,5));
          \end{asy}
        \end{center}
        \pause
      \column{.5\textwidth}
        \begin{itemize}
          \item $\triangle OAM\sim\triangle OPA$ \pause
          \item $\angle OAM=\angle DAP$ \pause
          \item $\angle BAD=\angle OAC$ \pause
          \item $\angle BAP=\angle MAC$ 
        \end{itemize}
    \end{columns}
  \end{frame}
  \begin{frame}[fragile]
    \frametitle{Symmedian}
    Let $ABC$ be a triangle with circumcircle $\Gamma$. Let the tangents to
    $\Gamma$ at $B$ and $C$ intersect at $P$, and let the midpoint of $BC$ be
    $M$. Prove that $\angle BAP=\angle MAC$.\pause

    Construction: let $X$ be the point such that $\triangle ABX$ is directly
    similar to $\triangle APC$. Let $M'$ be the intersection of $AX$ and $BC$.
    \begin{columns}
      \column{.5\textwidth}
        \begin{center}
          \begin{asy}
            pair A=(13,73)/1.1, B=(0,0), C=(100,0)/1.1;
            draw(circumcircle(A,B,C));
            pair O=circumcenter(A,B,C);
            pair M=(B+C)/2;
            pair P=length(B-O)*length(B-O)/(length(M-O)*length(M-O))*(M-O)+O;
            pair X=(B-A)*(C-A)/(P-A)+A;
            label("$A$",A,NW);
            label("$B$",B,SW);
            label("$C$",C,SE);
            label("$M'$",M,SE);
            label("$P$",P,E);
            label("$X$",X,S);
            draw(A--B--C--A--P--B--X--C--P);
            draw(A--M);
          \end{asy}
        \end{center}
        \pause
      \column{.5\textwidth}
        \begin{itemize}
          \item $\angle M'BA=\angle M'XB$ \pause
          \item $M'B^2=M'A\times M'X$ \pause
          \item $M'C^2=M'A\times M'X$ \pause
          \item $\angle BAP=\angle MAC$
        \end{itemize}
    \end{columns}
  \end{frame}
  \begin{frame}[fragile]
    \frametitle{Harmonic quad}
    Let $ABC$ be a triangle. Let $Q$ be a point such that the circumcircle of
    $AQB$ is tangent to $AC$ and the circumcircle of $AQC$ is tangent to $AB$.
    Let $D$ be the reflection of $A$ over $Q$. Prove that $ABCD$ is cyclic and
    $AB\times CD=BD\times AC$. \pause
    \begin{columns}
      \column{.5\textwidth}
        \begin{center}
          \begin{asy}
            pair A=(13,73)/1.1, B=(0,0), C=(100,0)/1.1;
            pair O=circumcenter(A,B,C);
            pair Y=B+C-A;
            pair Q=(B-A)*(C-A)/(Y-A)+A;
            pair D=2*Q-A;
            label("$A$",A,NW);
            label("$B$",B,SW);
            label("$C$",C,SE);
            label("$Q$",Q,E);
            label("$D$",D,SW);
            draw(Arc(circumcenter(A,Q,B),B,A));
            draw(Arc(circumcenter(A,Q,C),A,C));
            draw(D--C--Q--B--D--A--B--C--A);
            add(pathticks(A--Q));
            add(pathticks(Q--D));
          \end{asy}
        \end{center}
        \pause
      \column{.5\textwidth}
        \begin{itemize}
          \item $\triangle DQB\sim\triangle CQD$ \pause
          \item $\angle BAC+\angle BDC=180^\circ$ \pause
          \item $AB\times CD=BD\times AC$
        \end{itemize}
    \end{columns}
  \end{frame}
  \begin{frame}[fragile]
    \frametitle{Symmedian}
    Let $ABC$ be a triangle. Let $P$ be the intersection of the tangents from
    $B$ and $C$ to the circumcircle of $ABC$. 
    Let $Q$ be a point such that the circumcircle of
    $AQB$ is tangent to $AC$ and the circumcircle of $AQC$ is tangent to $AB$.
    
    Prove that $A, P, Q$ are collinear. \pause

    Constructions: define $D$ as previously. Let $O$ be the circumcentre of
    $ABC$.
    \begin{columns}
      \column{.5\textwidth}
        \begin{center}
          \begin{asy}
            pair A=(13,73)*0.7, B=(0,0), C=(100,0)*0.7;
            draw(circumcircle(A,B,C));
            pair O=circumcenter(A,B,C);
            pair Y=B+C-A;
            pair Q=(B-A)*(C-A)/(Y-A)+A;
            pair D=2*Q-A;
            pair M=(B+C)/2;
            pair P=(B-O)*(C-O)/(M-O)+O;
            label("$A$",A,NW);
            label("$B$",B,SW);
            label("$C$",C,SE);
            label("$Q$",Q,W);
            label("$D$",D,SW);
            label("$P$",P,E);
            label("$O$",O,N);
            draw(Arc(circumcenter(A,Q,B),B,A));
            draw(Arc(circumcenter(A,Q,C),A,C));
            add(pathticks(A--Q));
            add(pathticks(Q--D));
            draw(A--B--C--O--B--Q--C--D--B--P--C--A--D);
            draw(O--Q);
          \end{asy}
        \end{center}
        \pause
      \column{.5\textwidth}
        \begin{itemize}
          \item $\angle OQA=90^\circ$ \pause
          \item $BPCOQ$ cyclic \pause
          \item $A,Q,P$ collinear
        \end{itemize}
    \end{columns}
  \end{frame}
  \begin{frame}[fragile]
    \frametitle{Symmedian}
    Let $ABC$ be a triangle. Let $Q$ be a point such that the circumcircle of
    $AQB$ is tangent to $AC$ and the circumcircle of $AQC$ is tangent to $AB$.
    Let $M$ be the midpoint of $BC$.
    Prove that $\angle BAQ=\angle MAC$. \pause

    Construction: let $Y$ be the point such that $\triangle ABY$ is directly
    similar to $\triangle AQC$.
    \begin{columns}
      \column{.5\textwidth}
        \begin{center}
          \begin{asy}
            pair A=(13,73)/1.1, B=(0,0), C=(100,0)/1.1;
            draw(circumcircle(A,B,C));
            pair O=circumcenter(A,B,C);
            pair Y=B+C-A;
            pair Q=(B-A)*(C-A)/(Y-A)+A;
            label("$A$",A,NW);
            label("$B$",B,SW);
            label("$C$",C,SE);
            label("$Q$",Q,S);
            label("$Y$",Y,E);
            draw(Y--A--B--Y--C--Q--B--C--A--Q);
            draw(Arc(circumcenter(A,Q,B),B,A));
            draw(Arc(circumcenter(A,Q,C),A,C));
          \end{asy}
        \end{center}
        \pause
      \column{.5\textwidth}
        \begin{itemize}
          \item $\angle AYB=\angle YAC$ \pause
          \item $\angle AYC=\angle YAB$ \pause
          \item $A, M, Y$ collinear \pause
          \item $\angle BAQ=\angle MAC$
        \end{itemize}
    \end{columns}
  \end{frame}
  \begin{frame}[fragile]
    \frametitle{Butterfly}
    Let $ABCD$ be a cyclic quadrilateral with circumcentre $O$, and let $AC$ 
    and $BD$ intersect at $P$. A line through $P$ intersects $AB$ and $CD$ at
    $E$ and $F$, such that $OP$ is perpendicular to $EF$.

    Prove that $P$ is the midpoint of $EF$. \pause

    Constructions: let $M$ and $N$ be the midpoints of $AB$ and $CD$
    respectively.
    \begin{columns}
      \column{.5\textwidth}
        \begin{center}
          \begin{asy}
            pair A=(39,52), B=(39,-52), C=(-60,-25);
            pair O=(0,0);
            pair P=A+0.36*(C-A);
            pair D=intersectionpoint(B--(2*P-B),Circle(O,length(A)));
            label("$A$",A,N);
            label("$B$",B,SE);
            label("$C$",C,W);
            label("$D$",D,NW);
            draw(A--B--C--D--cycle);
            draw(A--C);
            draw(B--D);
            label("$P$",P,N);
            draw(Circle(O,length(A)));
            draw(O--P);
            pair Q=P*(0,1)+P;
            pair E=extension(P,Q,A,B), F=extension(P,Q,C,D);
            draw(E--F);
            label("$F$",F,NW);
            label("$O$",O,S);
            label("$E$",E,NE);
            pair M=(A+B)/2, N=(C+D)/2;
            draw(M--O--N--P--M);
            draw(E--O--F);
            label("$M$",M,SE);
            label("$N$",N,W);
          \end{asy}
        \end{center}
        \pause
      \column{.5\textwidth}
        \begin{itemize}
          \item $\triangle DPN\sim\triangle APM$ \pause
          \item $\triangle FOP\cong\triangle EOP$
        \end{itemize}
    \end{columns}
  \end{frame}
  \begin{frame}[fragile]
    \frametitle{Generalised Movie Theorem}
    Let $ABC$ and $PQR$ be directly similar triangles. If $X, Y, Z$ are points
    such that triangles $APX,\ BQY,\ CRZ$ are directly similar, then prove that
    $\triangle XYZ$ is also directly similar to triangles $ABC$ and $PQR$.
    \pause

    Construction: let $O$ be a point such that $\triangle OBC\overset
    +\sim\triangle OQR$.
    \begin{columns}
      \column{.5\textwidth}
        \begin{center}
          \begin{asy}
            unitsize(15);
            pair A=(0,0), B=(1.1,1.6), C=(0.76,4.14), P=(4.16,-0.68),
            Q=(4.98,2.44), X=(2.1,1.2);
            pair R=(Q-P)*(C-A)/(B-A)+P, Y=(Q-B)*(X-A)/(P-A)+B,
            Z=(Y-X)*(C-A)/(B-A)+X;
            pair M=extension(B,C,Y,Z);
            pair Nn=intersectionpoints(circumcircle(B,Y,M),
            circumcircle(C,Z,M))[1];
            draw(A--B--C--A, red);
            draw(P--Q--R--P, red);
            draw(X--Y--Z--X, red);
            draw(A--P--X--A, green);
            draw(B--Q--Y--B, green);
            draw(C--R--Z--C, green);
            draw(C--Nn--Z);
            draw(B--Nn--Y);
            label("$A$",A,SW);
            label("$B$",B,S);
            label("$C$",C,W);
            label("$P$",P,SE);
            label("$Q$",Q,E);
            label("$R$",R,N);
            label("$X$",X,E);
            label("$Y$",Y,NE);
            label("$Z$",Z,NE);
            label("$O$",Nn,W);
          \end{asy}
        \end{center}
        \pause
      \column{.5\textwidth}
        \begin{itemize}
          \item $OABC\sim OPQR$ \pause
          \item $OAPX\sim OBQY\sim OCRZ$ \pause
          \item $OABC\sim OXYZ\sim OPQR$ \pause
        \end{itemize}
        Note: $O$ is the centre of each of the \emph{spiral similarities}
        sending a coloured triangle to another of the same colour.
    \end{columns}
  \end{frame}
\end{document}
