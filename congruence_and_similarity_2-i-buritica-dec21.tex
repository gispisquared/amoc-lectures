\documentclass{article}
\usepackage[margin=2cm]{geometry}
\usepackage{amsmath}
\title{I --- Congruence and Similarity}
\author{Andres Buritica}
\date{December 8, 2021}
\begin{document}
\maketitle
\section{Intro}
  \subsection{Congruence}
    \begin{itemize}
      \item We say that triangles $ABC$ and $XYZ$ are \emph{congruent} if \[BC=YZ,\ CA=ZX,\
      AB=XY,\ \angle BAC=\angle YXZ,\]
      \[\angle ABC=\angle XYZ,\ \angle ACB=\angle XZY.\]

    \item If all of these conditions are true then we can write \[\triangle
      ABC\cong\triangle XYZ.\] We also know that $|ABC| = |XYZ|$.
      
    \item (SSS) If $BC=YZ,\ CA=ZX,\ AB=XY$ then $\triangle ABC\cong\triangle XYZ$.
    \item (SAS) If $AB=XY,\ \angle ABC=\angle XYZ,\ BC=YZ$ then $\triangle
      ABC\cong\triangle XYZ$.
    \item (AAS) If $\angle ABC=\angle XYZ,\ \angle BCA=\angle YZX,\ BC=YZ$ then $\triangle
      ABC\cong\triangle XYZ$.
    \item (SSA doesn't work) If $AB=XY,\ BC=YZ,\ \angle BCA=\angle YZX$ then you don't necessarily know
      $\triangle ABC\cong\triangle XYZ$.
    \item (fixed SSA) If $AB=XY,\ BC=YZ,\ \angle BCA=\angle YZX$ \textbf{and $AB>BC$} then
      $\triangle ABC\cong\triangle XYZ$.
    \item (RHS) If $AB=XY,\ BC=YZ,\ \angle BCA=\angle YZX=90^\circ$ then $\triangle
      ABC\cong\triangle XYZ$.
  \end{itemize}
  \subsection{Similarity}
  \begin{itemize}
    \item We say that triangles $ABC$ and $XYZ$ are \emph{similar} if
      \[\frac{BC}{YZ}=\frac{CA}{ZX}=\frac{AB}{XY}=r,\]
      \[\angle ABC=\angle XYZ,\
      \angle BCA=\angle YZX,\ \angle CAB=\angle ZXY.\]

    \item If all of these conditions are true then we can write \[\triangle
      ABC\sim\triangle XYZ.\] We also know that $\frac{|ABC|}{|XYZ|}=r^2$.

    \item Triangles can be directly similar ($\overset +\sim$) or oppositely
      similar ($\overset -\sim$).

    \item (PPP) If $\frac{BC}{YZ}=\frac{CA}{ZX}=\frac{AB}{XY}$ then $\triangle ABC\sim\triangle XYZ$.
    \item (PAP) If $\frac{AB}{XY}=\frac{BC}{YZ},\ \angle ABC=\angle XYZ$ then $\triangle
      ABC\sim\triangle XYZ$.
    \item (AA) If $\angle ABC=\angle XYZ,\ \angle BCA=\angle YZX$ then $\triangle
      ABC\sim\triangle XYZ$.
    \item (PPA doesn't work) If $\frac{AB}{XY}=\frac{BC}{YZ},\ \angle BCA=\angle YZX$ then you don't necessarily know
      $\triangle ABC\sim\triangle XYZ$.
    \item (fixed PPA) If $\frac{AB}{XY}=\frac{BC}{YZ},\ \angle BCA=\angle YZX$ \textbf{and $AB>BC$} then
      $\triangle ABC\sim\triangle XYZ$.
    \item (RHS) If $\frac{AB}{XY}=\frac{BC}{YZ},\ \angle BCA=\angle YZX=90^\circ$ then $\triangle
      ABC\sim\triangle XYZ$.
  \end{itemize}
\section{Lemmas}
All of these are ``well-known'' diagrams that you should learn to recognise as
subconfigurations of problems. However, the ideas used in their proofs are at
least as important as the configurations themselves.
\begin{enumerate}
  \item (Isosceles triangle) 
    Let $ABC$ be a triangle. Prove that $AB=AC$ if and only if $\angle
    ABC=\angle ACB$. 
  \item (Power of a point)
    Let $ABCD$ be a cyclic quadrilateral, and let $AC$ and $BD$ meet at $P$.
    Prove that $PA\times PC=PB\times PD$. 
  \item (Similar switch)
    Let $ABC$ and $ADE$ be triangles that are directly similar.
    Prove that $\triangle ABD$ and $\triangle ACE$ are also directly similar.
  \item (Ice cream cone theorem)
    Let $PA$ and $PB$ be tangents to a circle. Prove that $PA=PB$.
  \item (Similar figures)
    Let points $ABCDWXYZ$ be such that $\triangle ABC\overset +\sim\triangle WXY$ and
    $\triangle BCD\overset +\sim\triangle XYZ$. Prove that $\triangle
    ABD\sim\triangle WXZ$ and $\triangle ACD\sim\triangle
    WYZ$.

    In this case, we can say that $ABCD\overset +\sim WXYZ$.
  \item (Menelaus' Theorem) 
    Let $ABC$ be a triangle. Let $X,Y,Z$ be collinear points on sides
    $BC,CA$ and $AB$
    respectively. Prove that \[\frac{AZ}{ZB}\times\frac{BX}{XC}\times\frac{CY}{YX}=-1.\]
  \item (Homothetic triangles)
    Let $ABC$ and $XYZ$ be two triangles such that $BC||YZ,\ CA||ZX$ and $AB||XY$.
    Prove that $AX,\ BY$ and $CZ$ are concurrent.
  \item (Pythagoras' Theorem) 
    Let $ABC$ be a triangle such that $\angle BAC=90^\circ$. Prove that
    $AB^2+AC^2=BC^2$. 
  \item (Diagonals of a parallelogram)
    Let $ABCD$ be a parallelogram (that is, $AB||CD$ and $BC||DA$). Let $AC$
    intersect $BD$ at $P$. Prove that $P$ is the midpoint of $AC$ and of $BD$.
  \item (Alternate segment switch) 
    Let $A,B,C$ be collinear points, and let $D,E$ be points such that $AD=AE$.
    Prove that if $\angle ADB=\angle ACD$ then $\angle AEB=\angle ACE$. 
  \item (Ptolemy's Theorem) 
    Let $ABCD$ be points in that order around a circle. Prove that $AB\times
    CD+BC\times AD=AC\times BD$. 
  \item (Diameter of the incircle)
    Let $ABC$ be a triangle with incentre $I$ and $A$-excentre $I_A$. Let the
    incircle touch $BC$ at $D$ and the $A$-excircle touch $BC$ at $E$. Let $P$
    be the reflection of $D$ over $I$. Prove that $P$ is on $AE$. 
  \item (Symmedian)
    Let $ABC$ be a triangle with circumcircle $\Gamma$. Let the tangents to
    $\Gamma$ at $B$ and $C$ intersect at $P$, and let the midpoint of $BC$ be
    $M$. Prove that $\angle MAB=\angle PAC$.
  \item (Harmonic quadrilateral)
    Let $ABC$ be a triangle. Let $Q$ be a point such that the circumcircle of
    $AQB$ is tangent to $AC$ and the circumcircle of $AQC$ is tangent to $AB$.
    Let $D$ be the reflection of $A$ over $Q$. Prove that $ABCD$ is cyclic and
    $AB\times CD=BD\times AC$. 
  \item (Symmedian)
    Let $ABC$ be a triangle. Let $P$ be the intersection of the tangents from
    $B$ and $C$ to the circumcircle of $ABC$. 
    Let $Q$ be a point such that the circumcircle of
    $AQB$ is tangent to $AC$ and the circumcircle of $AQC$ is tangent to $AB$.
    Prove that $A, P, Q$ are collinear. 
  \item (Symmedian)
    Let $ABC$ be a triangle. Let $Q$ be a point such that the circumcircle of
    $AQB$ is tangent to $AC$ and the circumcircle of $AQC$ is tangent to $AB$.
    Let $M$ be the midpoint of $BC$.
    Prove that $\angle BAQ=\angle MAC$. 
  \item (Butterfly Theorem)
    Let $ABCD$ be a cyclic quadrilateral with circumcentre $O$, and let $AC$ 
    and $BD$ intersect at $P$. A line through $P$ intersects $AB$ and $CD$ at
    $E$ and $F$, such that $OP$ is perpendicular to $EF$. Prove that $P$ is the midpoint of $EF$. 
  \item (Generalised Movie Theorem)
    Let $ABC$ and $PQR$ be directly similar triangles. If $X, Y, Z$ are points
    such that triangles $APX,\ BQY$ and $CRZ$ are directly similar, then prove that
    $\triangle XYZ$ is also directly similar to triangles $ABC$ and $PQR$.
\end{enumerate}
\section{Problems}
Many of these are stolen from PST, but if you're inters you can't have done
that much of PST so it's fine right?
\begin{enumerate}
  \item Let $ABC$ be a triangle with circumcentre $O$, orthocentre $H$, centroid
    $G$ and nine-point centre $N$. Prove that $H,G,N,O$ are collinear with
    $OG:GN:NH=2:1:3$.
  \item Let $ABC$ be a triangle with $AB<AC$, and let $P$ be a point on segment
    $AC$ such that $AB=CP$. Prove that the perpendicular bisectors of $BC$ and
    $AP$ meet on the circumcircle of $ABC$.
  \item Let $ABCDE$ be a convex pentagon such that $\angle BAC=\angle CAD=\angle
    DAE$ and $\angle ABC=\angle ACD=\angle ADE$. The diagonals $BD$ and $CE$
    meet at $P$. Prove that the line $AP$ passes through the midpoint of $CD$.
  \item Let $ABC$ be a triangle with orthocentre $H$. Prove that the
    circumcentres of triangles $BCH,\ CAH$ and $ABH$ form a triangle congruent
    to $\triangle ABC$.
  \item Let $ABC$ be a triangle and let $P$ be a point on line $BC$. Let $Q$ and
    $R$ be points on lines $CA$ and $AB$ respectively, such that $QP=QC$ and
    $RP=RB$. Prove that the circumcircle of $\triangle AQR$ passes through the
    circumcentre of $\triangle ABC$.
  \item Let $ABC$ be a triangle with circumcentre $O$ and incentre $I$. Let $R$
    and $r$ be the circumradius and inradius of $\triangle ABC$ respectively. Prove that $R^2-OI^2=2Rr$.
  \item Let $P$ be a point inside triangle $ABC$ such that $\angle PBA=\angle
    PCA$. Let $Q$ and $R$ be the feet of
    the perpendiculars from $P$ to sides $AB$ and $AC$, and let $M$ be the
    midpoint of $AB$. Prove that $MQ=MR$.
  \item Let $P$ and $Q$ be on segment $BC$ of an acute triangle $ABC$ such that
    $\angle PAB=\angle BCA$ and $\angle CAQ=\angle ABC$. Let $M$ an $N$ be the
    points on $AP$ and $AQ$, respectively, such that $P$ is the midpoint of $AN$
    and $Q$ is the midpoint of $AN$. Prove that the intersection of $BM$ and
    $CN$ is on the circumcircle of triangle $ABC$.
  \item Let $ABC$ be a triangle, and let points $M$ and $N$ be on rays $AB$ and
    $AC$ respectively such that $AM=AN=BC$. Let $O$ be the point such that
    $AMON$ is a parallelogram. Prove that if $O$ is the $A$-excentre of triangle
    $ABC$, then $\triangle ABC$ is isosceles.
\end{enumerate}
\end{document}
